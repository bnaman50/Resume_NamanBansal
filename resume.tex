%%%%%%%%%%%%%%%%%%%%%%%%%%%%%%%%%%%%%%%%%%%%%%%%%%%%%%%%%%%%%%%%%%%%%%%%%%%%%%%%
% Medium Length Graduate Curriculum Vitae
% LaTeX Template
% Version 1.2 (3/28/15)
%
% This template has been downloaded from:
% http://www.LaTeXTemplates.com
%
% Original author:
% Rensselaer Polytechnic Institute 
% (http://www.rpi.edu/dept/arc/training/latex/resumes/)
%
% Modified by:
% Daniel L Marks <xleafr@gmail.com> 3/28/2015
% 
% Further modified by:
% Rohan Bavishi <rohan.bavishi95@gmail.com> 9/20/2016
%
% Important note:
% This template requires the simple_style.cls file to be in the same directory 
% as the .tex file. The res.cls file provides the resume style used for 
% structuring the document.
%
%%%%%%%%%%%%%%%%%%%%%%%%%%%%%%%%%%%%%%%%%%%%%%%%%%%%%%%%%%%%%%%%%%%%%%%%%%%%%%%%

%-------------------------------------------------------------------------------
%	PACKAGES AND OTHER DOCUMENT CONFIGURATIONS
%-------------------------------------------------------------------------------

%%%%%%%%%%%%%%%%%%%%%%%%%%%%%%%%%%%%%%%%%%%%%%%%%%%%%%%%%%%%%%%%%%%%%%%%%%%%%%%%
% You can have multiple style options the legal options ones are:
%
%   centered:	the name and address are centered at the top of the page 
%				(default)
%
%   line:		the name is the left with a horizontal line then the address to
%				the right
%
%   overlapped:	the section titles overlap the body text (default)
%
%   margin:		the section titles are to the left of the body text
%		
%   11pt:		use 11 point fonts instead of 10 point fonts
%
%   12pt:		use 12 point fonts instead of 10 point fonts
%
%%%%%%%%%%%%%%%%%%%%%%%%%%%%%%%%%%%%%%%%%%%%%%%%%%%%%%%%%%%%%%%%%%%%%%%%%%%%%%%%

\documentclass[mm]{simple_style}  

% Default font is the helvetica postscript font
\usepackage{helvet}
\usepackage{hyperref}
\usepackage{url}
\usepackage{xcolor}
\usepackage{scrextend}
\usepackage{fontawesome}

\usepackage{lmodern}
\usepackage[T1]{fontenc}

%% For enumeration
\usepackage{enumitem}

\hypersetup {
    colorlinks=true,
    linkcolor=colorlink,
    filecolor=magenta,      
    urlcolor=colorlink,
}
\usepackage[left=0.7in, right=2in, top=0.7in, bottom = 0.7in]{geometry}

% Increase text height
\textheight=700pt
\title{Resume_NamanBansal}
\newcommand\tab[1][1cm]{\hspace*{#1}}


\begin{document}

%-------------------------------------------------------------------------------
%	NAME AND ADDRESS SECTION
%-------------------------------------------------------------------------------

\name{{Naman Bansal}\\[-1.5ex]}
\qualification{PhD Student, CSSE Department, Auburn University\\[0pt]}

\emailtwo{nzb0040@auburn.edu}
\emailone{bnaman50@gmail.com}
\website{https://bnaman50.github.io/}{\textbf{bnaman50.github.io/}}
\github{https://github.com/bnaman50}{\textbf{github.com/bnaman50}}
\linkedin{https://www.linkedin.com/in/bnaman}{\textbf{linkedin.com/in/bnaman}}

%% If you want this, uncomment from class file as well
%\scholar{https://scholar.google.co.in/citations?hl=en\&user=naeglEQAAAAJ}{\textbf{google.scholar.com/bnaman}}
\phone{\cusemph{+1 334-524-8160}}
\address{}
%-------------------------------------------------------------------------------
%\vspace{5ex}
\begin{resume}

\iffalse
%-------------------------------------------------------------------------------
%	Biography
%-------------------------------------------------------------------------------
\vspace{-1ex}
\section{Biography}
\cusemph{Indian Institute of Technology Gandhinagar}, Gujarat, India\\
{\sl B.Tech.}, Electrical Engineering, \timeline{Jul' 12 - May' 16}
\\
\cusemph{GPA: 8.55/10}
\\[1ex]
%\vspace{-2ex}
\sectionline
%-------------------------------------------------------------------------------
\fi

%-------------------------------------------------------------------------------
%	EDUCATION SECTION
%-------------------------------------------------------------------------------
\vspace{-1ex}
\section{Education}
\cusemph{Auburn University}, Alabama, USA \timeline{Aug' 18 - Present}\\
{\sl M.S. \& Ph.D.}, Computer Science and Software Engineering, {\cusemph{GPA: 4/4}} 
\\[2ex] 
\cusemph{Indian Institute of Technology Gandhinagar}, Gujarat, India \timeline{July 12 - May 16}\\
{\sl B.Tech.}, Electrical Engineering, \cusemph{GPA: 8.55/10}
\\[1ex]
%\vspace{-2ex}
\sectionline
%-------------------------------------------------------------------------------

%-------------------------------------------------------------------------------
%	RESEARCH SECTION
%-------------------------------------------------------------------------------
\vspace{-1ex}
\section{Research\\Interests}
\par
Computer Vision, Deep Learning, Explainable Artificial Intelligence

%-------------------------------------------------------------------------------
%      PUBLICATIONS 
%-------------------------------------------------------------------------------
\vspace{1ex}
\section{Publications}

\cusemph{Bansal, Naman}*, Aggarwal, Chirag* and Nguyen, Anh*, \cusemph{The Sensitivity Of Attribution Methods To Hyperparameters} (CVPR 2020) (\textit{22\% acceptance rate})

\cusemph{Bansal, Naman} and Raman, Shanmuganathan, \cusemph{Regularized Tone Mapping Using Edge Preserving Filters}, in the \textit{21st National Conference on Communication}, IIT Bombay, IN, Feb. 27- Mar. 1, 2015. (\textit{30\% acceptance rate})
\\[1ex]
%\vspace{-2ex}
\sectionline
%-------------------------------------------------------------------------------

%-------------------------------------------------------------------------------
%       RESEARCH PROJECTS	
%-------------------------------------------------------------------------------
\vspace{-1ex}
\section{Research Projects}

\begin{project}
  \title{Explainable AI, Auburn University}
  \supervisor{Supervisor : Prof. Anh Nguyen, Artificial Intelligence Lab}
  \duration{Feb' - Nov' 19}
  \description{
    \vspace{-1.5ex}
    \begin{itemize}[label={\textbf{\textendash}}, leftmargin=2ex, topsep=0pt]
        \item Identified and addressed several issues with \textasciitilde 15 famous XAI papers
        \item Published an important criterion - \textit{A key desidratum of an explanation method is its robustness to input hyperparameters}
    \end{itemize}
	\iffalse
	- Our work contributed a key criterion - 'A key desidratum of an explanation method is its robustness to input hyperparameters' and has been accepted for CVPR 2020
	\fi
  } 
\end{project}
\vspace{-1.5ex}

\begin{project}
  \title{Classification of Imbalanced data using Deep Learning, University of Notre Dame}
  \supervisor{Supervisor : Prof. Nitesh Chawla, Director iCeNSA}
  \duration{May - July 15}
  \description{
    \vspace{-1.5ex}
    \begin{itemize}[label={\textbf{\textendash}}, leftmargin=2ex, topsep=0pt]
        \item Devised and programmed a solution to the class imbalance problem using stacked de-noising auto-encoders and pre-processing techniques such as SMOTE and Tomek links
        \item Achieved comparable F-scores to classical techniques such as CART and Random Forest 
    \end{itemize}
 % - 
 % Developed a thorough understanding of the machine learning algorithms such as % Linear Regression, Random Forests, SVM and neural networks\\
  } 
\end{project}
\vspace{-1.5ex}

\begin{project}
\title{TMO for HDR Imaging, IIT Gandhinagar}
\supervisor{Supervisor : Prof. Shanmuganathan Raman, Computer Vision Lab}
\duration{Aug' - Nov' 14}
\description{
    \vspace{-1.5ex}
    \begin{itemize}[label={\textbf{\textendash}}, leftmargin=2ex, topsep=0pt]
        \item Proposed a novel \href{https://github.com/champnaman/ToneMapping}{Tone Mapping Operator} (TMO) for High Dynamic Range (HDR) Images % Imaging problem
        \item Implemented an iterative solution to preserve the contrast and minute details present in input HDR image using an edge-preserving filter %(bilateral filter or WLS filter) 
    \end{itemize}
}
\end{project}
\vspace{-1.5ex}

\begin{project}
  \title{Time Delay Integration (TDI) Imaging, Space Application Centre, ISRO }
  \supervisor{Supervisor : Mr. Ashish Mishra, Head PCSVD}
  \duration{May - July 14}
  \description{
    \vspace{-1.5ex}
    \begin{itemize}[label={\textbf{\textendash}}, leftmargin=2ex, topsep=0pt]
        \item Simulated and analyzed the effects of non-linear platform characteristics  in TDI Imaging %and optical butting in TDI Imaging
        \item Quantified the distortions based on signal to noise ratio and modular transfer function
        \item Integrated the entire work in a simple GUI using Matlab GUIDE
    \end{itemize}
}
\end{project} 
\\[0ex]
\sectionline
%-------------------------------------------------------------------------------

%-------------------------------------------------------------------------------
%	ACADEMIC PROJECTS SECTION
%-------------------------------------------------------------------------------
\vspace{-1ex}
\section{Academic\\Projects}
\begin{project}
\title{Copy-Move Forgery Detection, IIT Gandhinagar}
\supervisor{Supervisor : Prof. Nitin Khanna}
\duration{Aug' - Nov' 15}
\description{
    \vspace{-1.5ex}
    \begin{itemize}[label={\textbf{\textendash}}, leftmargin=2ex, topsep=0pt]
        \item Investigated \href{https://www.dropbox.com/s/99aktrn93bf7zkj/study-copy-move.pdf?dl=0}{block-based methods} for copy-move forgery detection such as Zernike Moments, SVD, PCA, and KPCA
    \end{itemize}
}
\end{project}
\vspace{-1.5ex}

\begin{project}
\title{Non-Intrusive Load Modelling, IIT Gandhinagar}
\supervisor{Supervisor : Prof. Babji Srinivasan and Prof. Rajagopalan Srinivasan}
\duration{Jan' - Mar' 15}
\description{
    \vspace{-1.5ex}
    \begin{itemize}[label={\textbf{\textendash}}, leftmargin=2ex, topsep=0pt]
        \item Studied non-intrusive load monitoring models for electricity consumption in buildings
        \item Programmed a \href{https://github.com/champnaman/EniMeter_Data_Read}{script} to collect electricity consumption data using EniMeter SCPM - T12 
    \end{itemize}
}
\end{project}
\vspace{-1.5ex}
%\newpage

%% Comments
\iffalse
\begin{project}
\title{Low Cost X-Ray Detector, IIT Bombay}
\supervisor{Supervisor : Achuta Kadambi, MIT Media Lab and Dr. Rajiv Gupta, MGH}
\duration{24 - 31 Jan '14}
\description{
- Came up with the idea of using Selfoc lens array system present in flatbed scanner for the design of low-cost
X-ray detector 
}
\end{project} 
\vspace{-2ex}
\fi

\iffalse
\begin{project}
\title{Hand-Written Digit Recognition using PCA, IIT Gandhinagar}
\supervisor{Supervisor : Babji Srinivasan}
\duration{Nov '16}
\description{
- Implemented a simple MNIST classifier using PCA
}
\end{project} 
\vspace{-2ex}
\fi

\begin{project}
\title{Database Management System, IIT Gandhinagar}
\supervisor{Supervisor : Prof. Gaurav Srivastava}
\duration{Nov' - Dec '13}
\description{
- Programmed a \href{https://github.com/champnaman/BasicDB_Sys}{system} for maintenance of IITGN student database using SQLite
}
\end{project}
\\[-1.6ex]
\sectionline
%-------------------------------------------------------------------------------

%-------------------------------------------------------------------------------
%	Academic Highlights
%-------------------------------------------------------------------------------
\vspace{-1ex}
\section{Academic Highlights}
\vspace{9pt}
\begin{itemize}[label={}, leftmargin=0pt, topsep=0pt]
\item Awarded \cusemph{Graduate Teaching Assistantship} by Auburn University (Aug' 18 - Present)
%comments
\iffalse
\item Recipient of Merit-cum-Means scholarship award for meritorious performance throughout the undergraduate program
\item Featured thrice in \cusemph{Dean's List} Award for Academic Excellence (IITGN) ()
\fi
\item Three time awardee of \cusemph{\href{https://www.iitgn.ac.in/students/deanslist}{Dean's List}} Award for Academic Excellence (IITGN) (July 12 - May 16)
\item Recipient of the \cusemph{\href{https://www.ee.iitb.ac.in/~ncc2015/travelfellowships.html}{National Instruments Student Travel Fellowship}} for NCC 2015 conference
\item Secured an \cusemph{All-India-Rank} of 4224 in Joint Entrance Examination 2012 amongst \textasciitilde half a million candidates
\item Recipient of scholarship under \cusemph{Central Sector Scholarship Scheme} (CSSS) for outstanding performance in XII Exams
\end{itemize}
\vspace{-2ex}
\sectionline
%-------------------------------------------------------------------------------

%-------------------------------------------------------------------------------
%	COMPUTER SKILLS SECTION
%-------------------------------------------------------------------------------
\vspace{-1ex}
\section{Technical\\Skills}
% \vspace{9pt}
\begin{itemize}[label={}, leftmargin=0pt, topsep=0pt]
    \item \begin{tabular}{@{\hskip -0.2pt}l @{\hskip 1ex}l }
    \cusemph{Languages:} & Python, MATLAB (\textit{Most Experienced}) \\
                         & C, C++, HTML5, CSS3, JavaScript, PHP, \LaTeX (\textit{Basic})\\
    \end{tabular}
    \item \cusemph{Libraries}: TensorFlow, OpenCV, scikit, Keras, PyTorch 
    \item \cusemph{Operating Systems}: Windows, MacOS, Linux
\end{itemize}
\vspace{-2ex}
\sectionline
%-------------------------------------------------------------------------------

%-------------------------------------------------------------------------------
%	RELEVANT COURSES
%-------------------------------------------------------------------------------
\vspace{-1ex}
\section{Relevant Courses}
\begingroup % to strectch the space among the cols
    \renewcommand{\arraystretch}{1.1} % Default value: 1
    \begin{tabular}{@{\hskip -0.2pt}l @{\hskip 15ex}l }
        Deep Learning (Auburn University) & Machine Learning (Auburn University) \\
        Signals and Systems (IITGN) & Probability and Random Processes (IITGN) \\
        3D Computer Vision (IITGN) & Applied Multivariate Data Analysis (IITGN)\\
        Linear Algebra (MIT OCW) & Algorithms, Part I - Stanford (Coursera)\\
    \end{tabular}
\endgroup
\\[2.5ex]
\sectionline
%-------------------------------------------------------------------------------

%-------------------------------------------------------------------------------
%	TEACHING/MENTORING
%-------------------------------------------------------------------------------
\vspace{-1ex}
\section{Co-Curri-cular Activities}
\vspace{9pt}
\begin{itemize}[label={}, leftmargin=0pt, topsep=0pt]
    \item \cusemph{Graduate Teaching Assistant} \timeline{Aug' 18 - Present}
        \begin{itemize}[label={}, topsep=3pt]
            \item Leads a lab of \textasciitilde 15 students as a part of \cusemph{Intro to MATLAB Programming} course
        \iffalse
            \item Currently a GTA for \cusemph{Intro to MATLAB Programming} course conducted by Prof. Jacqueline Hundley
            \item Also lead a lab of \textasciitilde 15 students helping them understand the concepts of MATLAB programming
        \fi
        \end{itemize}
        
    \item \cusemph{Student Run Course} \timeline{Aug' - Nov' 15}
        \begin{itemize}[label={}, topsep=3pt]
            \item Instructed \iffalse Developed and instructed \fi a course on \cusemph{Applied Computer Programming} to a class of \textasciitilde 30 students
            \item Explained the concepts of \cusemph{curve fitting, interpolation, solution of linear systems of equations} using MATLAB and Python
        \end{itemize}
        
    \item \cusemph{Course Assistant under EWYL} \timeline{Aug' 14 - May 15}
        \begin{itemize}[label={}, topsep=3pt]
            \item Selected for \cusemph{Earn While You Learn (EWYL)} program by IITGN aimed at helping students get first-hand professional work experience with accountability 
            \iffalse
            \item Assisted Prof. Mohan Joshi for his course on \cusemph{Mathematical Methods in Engineering} 
            \fi
            \item Developed modules for redundancy elimination in price indices of various commodities and a Progressive and Iterative approximation (PIA) approach for curve fitting  
        \end{itemize}
        
    \item \cusemph{\href{https://legacy.iitgn.ac.in/counseling.htm}{Student Guide} for Freshmen Students} \timeline{Aug' 14 - Aug' 15}
        \begin{itemize}[label={}, topsep=3pt]
            \item Served as the first point of contact for freshmen in helping them settle in new environment
        \end{itemize}
        
    \item \cusemph{\href{http://sites.iitgn.ac.in/explorer/}{IIT Gandhinagar Explorer's Fellow}} \timeline{May - July 16}
        \begin{itemize}[label={}, topsep=3pt]
            \item Spent \cusemph{\href{http://sites.iitgn.ac.in/explorer/2016-explorer/teamnavigatio/}{seven weeks backpacking}} across India at a meager budget of INR $800$ per day 
        \end{itemize}
        
    \item \cusemph{Badminton}
        \begin{itemize}[label={}, topsep=3pt]
            \item Represented the institute in \cusemph{\href{http://iitb.ac.in/en/event/50th-inter-iit-sports-meet-2014}{50th Inter IIT Sports Meet}}, IIT Bombay (Dec 2014)
        \end{itemize}
        
\end{itemize}

%% comments
\iffalse
\begin{description}
\item[Teaching Assistant]\ \\ 
- TA for \cusemph{Intro to MATLAB Programming} course conducted by Prof. Jacqueline Hundley\\
- Leads a lab for a class of approx. 15 students
\item[Student Run Course]\ \\
- Instructor of a \cusemph{student-run course titled Applied Computer Programming}\\ %during senior year\\
- Covered the concepts of \cusemph{curve fitting, interpolation, solution of linear systems of equations} using MATLAB and Python
\item[Teaching Assistant]\ \\
- TA for \cusemph{Mathematical methods for engineers} course conducted by Prof. Mohan Joshi\\
- Developed modules for redundancy elimination in price indices of various commodities and a Progressive and Iterative approximation (PIA) approach for curve fitting  
\item[Student Guide for Freshmen Students]\ \\
- Served as the first point of contact for freshmen in helping them settle in new environment\\
- Selection process involved evaluation of ethics and emotional stability of the applicant 
\end{description}
\fi
%% till here




%-------------------------------------------------------------------------------

%-------------------------------------------------------------------------------
%	Extra-Curricular Actives
%-------------------------------------------------------------------------------
\iffalse
\section{Extra-Curricular Activities}
\begin{description}
\item[IIT Gandhinagar Explorer's Fellow]\ \\
- Recipient of \cusemph{\href{http://sites.iitgn.ac.in/explorers/2016-explorers/teamnavigatio/}{Explorer Fellowship}}, unique to our institute\\
- Spent \cusemph{seven weeks backpacking} across
India at a meager budget of INR $800$ per day
\item[Badminton]\ \\
- Member of \cusemph{IIT Gandhinagar's Badminton team} in Sophomore and Junior year\\ 
- Represented the institute in \cusemph{50th Inter IIT Sports Meet}, IIT Bombay (Dec 2014)
\iffalse
\item[Event Marketing]\ \\
- Marketing team member of institute's \cusemph{Annual Technical Summit, \href{http://amalthea.iitgn.ac.in/2012/}{Amalthea 2012}}\\
- Helped increase the event footfall from 2000 to 6000 
\item[Drama Club]\ \\
- An active member of institute's drama club \cusemph{Abhinaya} in Freshmen and Sophomore year   
\fi
\end{description}
\fi

%-------------------------------------------------------------------------------
\end{resume}
\end{document}
